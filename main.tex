\documentclass[11pt]{article}

\usepackage{hyperref}
\usepackage{xcolor}
\usepackage{calc}
\usepackage{graphicx}
\usepackage{tikz}
\usepackage{fontspec}
\usepackage{fontawesome5}
\usepackage{titlesec}
\usepackage{enumitem}
\usepackage{fancybox}
\usepackage{fancyhdr}
\usepackage{eso-pic}
\usepackage{tabularx}
\usepackage{array}
\usepackage[export]{adjustbox}

\hypersetup{hidelinks}

%%%%%%%%%%%%%%%%%%%%
% 设置
%%%%%%%%%%%%%%%%%%%%

\setlength{\parindent}{0pt}					% 取消全局段落缩进
\pagenumbering{gobble}						% 取消页码显示
\setlist[itemize]{nosep                     % 取消 itemize 的默认间距
    , before={\vspace*{-\parskip}}          % 取消 itemize 和后续段落之间的空白
    , leftmargin=*}		                    % 取消 itemize 的左边距
\setlist[enumerate]{leftmargin=*}	        % 取消 enumerate 的左边距
\renewcommand{\arraystretch}{1.2}           % 设置表格行间距
\linespread{1.25}                           % 设置正文行间距

\titleformat{\section}					    % 将原标题前面的数字取消了
  {\LARGE\bfseries\raggedright} 		      % 字体改为 LARGE,bold,左对齐
  {}{0em}                      			  % 可用于添加全局标题前缀
  {}                           			  % 可用于添加代码
  [{\color{secondary_color}\titlerule}]     % 标题下方加一条线
\titlespacing*{\section}{0cm}{*1.2}{*1.2}	% 标题左边留白,上方,下方

% 修正:增加 headheight 避免尺寸警告
\usepackage[
	a4paper,
	left=1.2cm,
	right=1.2cm,
	top=1.5cm,
	bottom=1cm,
	headheight=2.5cm, % 增加页眉高度
	footskip=1.5cm,   % 增加页脚偏移
	nohead
]{geometry}                                 % 页面边距设置

% 字体设置
\setmainfont[
    Path=fonts/,
    Extension=.otf,
    BoldFont=*-Bold,
]{NotoSerifSC}

\newlength{\iconwidth}
\setlength{\iconwidth}{1.5em}                   % 设置 section 标题部分图标占用的宽度

%%%%%%%%%%%%%%%%%%%%
% 自定义命令
%%%%%%%%%%%%%%%%%%%%

% 学院
\newcommand{\school}{网络与信息安全学院 | School of Cyber Engineering}

% 联系方式
\newcommand{\contact}{
    \footnotesize
    \textcolor{white}{
        % 邮箱
        \faEnvelope \quad \href{mailto:youremail@xidian.edu.cn}{youremail@xidian.edu.cn}
        \hspace{4em}
        % 手机号
        \faPhone \quad  130-6666-0000
        \hspace{4em}
        % GitHub
        \faGithub \quad \href{https://github.com/imamtom/XDU_CV}{GitHub 项目地址}
    }
}

% 主题切换命令
\newcommand{\setThemeRed}{
    \definecolor{primary_color}{RGB}{220, 38, 38}
    \definecolor{secondary_color}{RGB}{239, 68, 68}
    % 同时设置页眉页脚的图片
    \newcommand{\headerImage}{images/header_red.png}
    \newcommand{\footerImage}{images/footer_red.png}
    \newcommand{\backgroundImage}{images/xdu_logo_red.png}
}
\newcommand{\setThemeBlue}{
    \definecolor{primary_color}{RGB}{37, 99, 235}
    \definecolor{secondary_color}{RGB}{59, 130, 246}
    % 同时设置页眉页脚的图片
    \newcommand{\headerImage}{images/header_blue.png}
    \newcommand{\footerImage}{images/footer_blue.png}
    \newcommand{\backgroundImage}{images/xdu_logo_blue.png}
}

% 修正页眉页脚设置
\pagestyle{fancy}
\fancyhf{} % 清空默认


% \AddToShipoutPictureBG 作用范围:所有后续页面
% \AddToShipoutPictureBG* 作用范围:当前页面

\AddToShipoutPictureBG{
    % Placing the header image at the top
    \begin{tikzpicture}[remember picture, overlay]
        \node at (current page.north west) [
        anchor=north west,
        inner sep=0pt
        ](header) {
        \includegraphics[width=\paperwidth]{\headerImage}
        };
        \node[anchor=west](school_logo) at (header.west){
            \hspace{0.5cm}
            \includegraphics[width=0.24\textwidth]{images/xdu_logo_name_white.png}
        };
        \node[anchor=east](school_name) at(header.east){
            \textcolor{white}{\textbf{\school}}
            \hspace{0.5cm}
        };
    \end{tikzpicture}
    
    % Placing the footer image at the bottom
    \begin{tikzpicture}[remember picture, overlay]
        \node at (current page.south west) [
        anchor=south west,
        inner sep=0pt
        ](footer) {
        \includegraphics[width=\paperwidth]{\footerImage}
        };
        % 联系方式,不再以页脚的方式存在
        % \node[anchor=center] at(footer.center){\contact};
    \end{tikzpicture}
    
    % Placing the semi-transparent logo in the center
    \begin{tikzpicture}[remember picture, overlay]
        \node[opacity=0.15] at ([xshift=0cm,yshift=-1cm,]current page.center) {
        \includegraphics[width=0.7\paperwidth,keepaspectratio]{\backgroundImage}
        };
    \end{tikzpicture}
}


\begin{document}

% 设置主题颜色
\setThemeRed
% \setThemeBlue

% 第一页由于backgroundImage的原因,需要被裁去
need to cut off

\newpage
%%%%%%%%%%%%%%%%%%%%
% 简历正文
%%%%%%%%%%%%%%%%%%%%
    
    % % 个人信息
    % \begin{minipage}[t]{\textwidth}
        
    %     % \faGraduationCap这类\fa开头的都是font awesome里的logo,想换成其他logo的话
    %     可以看一下https://mirror-hk.koddos.net/CTAN/fonts/fontawesome5/doc/fontawesome5.pdf
    %     https://fontawesome.com/
    %         % 左边,照片,比例占行宽20

        
    % \begin{minipage}[t]{0.15\textwidth}
    %     % 左侧照片
    %     % \centering
    %     \includegraphics[width=0.95\linewidth]{images/kun.png}
    % \end{minipage}
    % \hfill

    % \end{minipage}
    \begin{tabularx}{\textwidth}{@{}m{0.15\textwidth}>{\hspace{0em}}X@{}}
        \vspace{2em}
        \adjustbox{frame,margin=2pt}{\includegraphics[width=0.95\linewidth]{images/kun.png}}
        &
        % 右侧内容(用minipage或直接写)
        \begin{minipage}[t]{\linewidth}
            \begin{minipage}[t]{0.32\textwidth}
                % 第一列第一行
                \textbf{姓名}:ikun
            \end{minipage}
            \hfill
            \begin{minipage}[t]{0.32\textwidth}
                % 第二列第一行
                \textbf{出生年月}:2000.01
            \end{minipage}
            \hfill
            \begin{minipage}[t]{0.32\textwidth}
                % 第三列第一行
                \textbf{电话}:133-3333-3333
            \end{minipage}
            
            \vspace{1em}
            
            \begin{minipage}[t]{0.32\textwidth}
                % 第一列第二行
                \textbf{性别}:鸡你太美
            \end{minipage}
            \hfill
            \begin{minipage}[t]{0.32\textwidth}
                % 第二列第二行
                \textbf{政治面貌}:群众
            \end{minipage}
            \hfill
            \begin{minipage}[t]{0.32\textwidth}
                % 第三列第二行
                \textbf{邮箱}:ikun@xidian.edu.cn
            \end{minipage}
        \end{minipage}
    \end{tabularx}
    \vspace{0.1em}


    % 教育背景
    \begin{minipage}[t]{\textwidth}
        \section[教育背景]{\makebox[\iconwidth][c]{\color{primary_color}{\faGraduationCap}}\quad 教育背景}
        
        \vspace{0.5em}
        {\large \textbf{西安电子科技大学}},博士 \hfill 2021.09--2025.06
        \begin{itemize}
            \item 网络空间安全专业\ 网络与信息安全学院 \hfill 西安
            \item \textbf{研究方向}:LLM安全、联邦学习、隐私计算 \hfill 导师:张三\ 教授
        \end{itemize}

        \vspace{0.5em}
        {\large \textbf{西安电子科技大学}},硕士 \hfill 2018.09--2021.06
        \begin{itemize}
            \item 网络空间安全专业\ 网络与信息安全学院 \hfill 西安
            \item \textbf{研究方向} LLM安全、密码学、区块链 \hfill 导师:张三\ 教授
            \item \textbf{GPA}:4.8 / 4.8(排名:1 / 250)
        \end{itemize}

        \vspace{0.5em}
        {\large \textbf{西安电子科技大学}},本科 \hfill 2014.09--2018.06
        \begin{itemize}
            \item 网络空间安全专业\ 网络与信息安全学院 \hfill 西安
            % \item \textbf{主修课程}:课程1、课程2、课程3、课程4\ 等。
            \item \textbf{GPA}:4.8 / 4.8(排名:1 / 250)
        \end{itemize}

        \vspace{1.2em}
    \end{minipage}



    
    % 科研成果    
    \begin{minipage}[t]{\textwidth}
        \section[科研成果]{\makebox[\iconwidth][c]{\color{primary_color}{\faFile*[regular]}}\quad 科研成果}
        \begin{itemize}
            \item \textbf{Your name}, He Wangkai, Zhang san, et al.  A method of network attack-defense game and collaborative defense decision-making based on hierarchical multi-agent reinforcement learning[J]. Computers \& Security, 2024, 142: 103871. \hfill \textbf{唯一通信}(中科院一区)
            \vspace{0.5em}
            \item 授权发明专利 一种基于卷积神经网络的美食识别系统(ZL202520020202). \hfill \textbf{第三发明人}
        \end{itemize}
        
        \vspace{1.2em}
    \end{minipage}

    % 项目经历\科研经历\项目与教学(标题请根据需要修改)
    \begin{minipage}[t]{\textwidth}
        \section[科研项目]{\makebox[\iconwidth][c]{\color{primary_color}{\faChalkboardTeacher}}\quad 科研项目}
        
        {\large \textbf{国家自然科学基金面上项目“基于卷积神经网络的美食识别系统”(6356781)}} \hfill 2020年9月--2021年9月
        \begin{itemize}
            \item 作为项目组成员参与,负责某某模块的设计与实现。
        \end{itemize}

        {\large \textbf{国家重点研发计划基金资助项目“基于卷积神经网络的美食识别系统”(2025YF102020)}} \hfill 2020年9月--2021年9月
        \begin{itemize}
            \item 作为学生骨干参与,负责某某模块的设计与实现。
        \end{itemize}
        
        \vspace{1.2em}
    \end{minipage}
    
    % 技能竞赛
    \begin{minipage}[t]{\textwidth}
        \section[技能竞赛]{\makebox[\iconwidth][c]{\color{primary_color}{\faAtom}}\quad 技能竞赛}

        \begin{itemize}
            \item \textbf{竞赛名称1},获奖等级 \hfill 2020年9月
            \item \textbf{竞赛名称2},获奖等级 \hfill 2021年6月
            \item \textbf{竞赛名称3},获奖等级 \hfill 2022年3月
        \end{itemize}
        
    \vspace{1.2em}
    \end{minipage}

    % 荣誉奖项
    \begin{minipage}[t]{\textwidth}
        \section[荣誉奖项]{\makebox[\iconwidth][c]{\color{primary_color}{\faCrown}}\quad 荣誉奖项}
        

        \begin{itemize}
            \item 一等奖学金三次 \hfill 硕士期间
            \item 二等奖学金三次 \hfill 博士期间
            \item 优秀学生干部 \hfill 2021年6月
            \item 研究生国家奖学金 \hfill 2022年9月
        \end{itemize}
        
    \vspace{1.2em}
    \end{minipage}

    % 实习经历
    \begin{minipage}[t]{\textwidth}
        \section[实习经历]{\makebox[\iconwidth][c]{\color{primary_color}{\faChalkboardTeacher}}\quad 实习经历}
        
        {\large \textbf{蚂蚁金融}} \hfill 2020年8月--2021年9月
        \begin{itemize}
            \item \textbf{Java开发实习生} ,负责某某项目的开发工作。
        \end{itemize}
        \vspace{0.5em}
        {\large \textbf{华为}} \hfill 2022年8月--2023年9月
        \begin{itemize}
            \item \textbf{C++开发实习生},负责某某项目的开发工作。
        \end{itemize}

    \vspace{1.2em}
    \end{minipage}


    % 个人总结
    \begin{minipage}[t]{\textwidth}
        \section[个人总结]{\makebox[\iconwidth][c]{\color{primary_color}{\faUserCircle}}\quad 个人总结}
        
        \begin{itemize}
            \item 勤奋好学,具有较强的学习能力和适应能力。
            \item 具有良好的团队合作精神和沟通能力。
        \end{itemize}
        
    \vspace{1.2em}
    \end{minipage}


    % % 如果每行的内容不是很多,可以考虑使用 minipage,将内容分列展示
    % \begin{minipage}[t]{0.6\textwidth}
    %     \section[技能特长]{\makebox[\iconwidth][c]{\color{primary_color}{\faWrench}}\quad 技能特长}
    %     \begin{itemize}
    %     \setlength{\itemsep}{0.5em}
    %         \item 熟练使用 Python、Jvav、Rust 等编程语言。
    %         \item 熟练使用 Tensorflow、Pytorch 等深度学习框架。
    %         \item 熟悉 Windows 与 Linux 端开发。
    %     \end{itemize}
    % \end{minipage}
    % \hfill
    % \begin{minipage}[t]{0.35\textwidth}
    %     \section[兴趣爱好]{\makebox[\iconwidth][c]{\color{primary_color}{\faStar}}\quad 兴趣爱好}
    %     \begin{itemize}
    %     \setlength{\itemsep}{0.5em}
    %         \item 我喜欢唱
    %         \item 跳
    %         \item Rap
    %     \end{itemize}
    % \end{minipage}
    


\end{document}
